\documentclass{amsart}
\usepackage[utf8]{inputenc}
\usepackage[style=alphabetic]{biblatex}
\usepackage{babel}
\usepackage{hyperref}

\addbibresource{}
\bibliography{export}


\begin{document}

\title{D2P-SP22}

\author{Emma Gibbs}
%\email{emma.gibbs@ucdenver.edu}

\author{Michael T. Schmidt}
%\email{michael.t.schmidt@ucdenver.edu}

\author{Evan Shapiro}
%\email{evan.shapiro@ucdenver.edu}

%\begin{abstract}
%\end{abstract}
\maketitle

\section{Proposed Topics}

\subsection{Redistricting}
\subsubsection{Data}

\subsection{Maximal Effect of Proposed Transportation Line}
With the \emph{Fastracks} RTD initiative behind schedule for extending the B-line to Boulder and Longmont, would the all-pair maximal flow be sufficient, to warrant the cost, given the changes to the Denver regional transportation network since the initial assessment.

\subsubsection{Data}

\subsection{Shared-Streets Initiative: How street closures impact flow}
During the pandemic, Denver changed a number of policies regarding street usage in the city; most noticeably, several streets were converted to pedestrian malls for use by walkers, runners, bikers, and other non-motorized transportation.
As the pandemic has waned, several of these policies may be changed back.
Since there is an interest in preserving or extending such shared streets, which streets may be converted with the least impact on the all-pairs maximal flow within the Denver metro area.

\subsubsection{Data}


\nocite{*}

\printbibliography%
\end{document}
